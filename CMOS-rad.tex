\documentclass[conference]{IEEEtran}
% \IEEEoverridecommandlockouts
% The preceding line is only needed to identify funding in the first footnote. If that is unneeded, please comment it out.
\usepackage{cite}
\usepackage{amsmath,amssymb,amsfonts}
\usepackage{algorithmic}
\usepackage{graphicx}
\usepackage{textcomp}
\usepackage{xcolor}
\def\BibTeX{{\rm B\kern-.05em{\sc i\kern-.025em b}\kern-.08em
    T\kern-.1667em\lower.7ex\hbox{E}\kern-.125emX}}
\begin{document}

\title{Correction Techniques for Image Faults}

\author{\IEEEauthorblockN{1\textsuperscript{st} Dominic Gaiero}
\IEEEauthorblockA{\textit{Computer Engineering Department} \\
\textit{California Polytechnic State University, San Luis Obispo}\\
San Luis Obispo, USA \\
dgaiero@calpoly.edu}
\and
\IEEEauthorblockN{2\textsuperscript{nd} Given Name Surname}
\IEEEauthorblockA{\textit{Computer Engineering Department} \\
\textit{California Polytechnic State University, San Luis Obispo}\\
San Luis Obispo, USA \\
email address or ORCID}
}

\maketitle

\begin{abstract}
This document is a model and instructions for \LaTeX.
This and the IEEEtran.cls file define the components of your paper [title, text, heads, etc.]. *CRITICAL: Do Not Use Symbols, Special Characters, Footnotes, 
or Math in Paper Title or Abstract.
\end{abstract}

\begin{IEEEkeywords}
CMOS, CIS
\end{IEEEkeywords}

\section{Introduction}
In this project the effectiveness radiation hardening a device through software was tested.  This expands on past methods by having picture correction happen in the data processing phase, rather than redesigning from a hardware level. While using software to mitigate the effects of radiation on sensors is not a permanent solution, the benefits of significant reduced costs and development make this solution very appealing for less critical applications. This project targets uses in security at a radiation plant or non-critical cameras in space or other radiation intensive conditions. This will allow systems that use cameras in these conditions to operate with more reliability and allow for more clear images despite the conditions and operation of the cameras.  Also, this solution aims to give more information on the nature of a fault in a sensor to a user, unlike traditional image processing techniques.  


\begin{thebibliography}{00}
% http://sirad.pd.infn.it/scuola_legnaro_2009/Presentazioni_Web/18_ScuolaLNL2009_Gerardin_Web.pdf
\bibitem{b1}T. Watanabe, T. Takeuchi, O. Ozawa, H. Komanome, T. Akahori, K. Tsuchiya,  ``A new radiation hardened CMOS image sensor for nuclear plant,''  www.imagesensors.org, 2017.
\bibitem{b2} J. Clerk Maxwell, A Treatise on Electricity and Magnetism, 3rd ed., vol. 2. Oxford: Clarendon, 1892, pp.68--73.
\bibitem{b3} I. S. Jacobs and C. P. Bean, ``Fine particles, thin films and exchange anisotropy,'' in Magnetism, vol. III, G. T. Rado and H. Suhl, Eds. New York: Academic, 1963, pp. 271--350.
\bibitem{b4} K. Elissa, ``Title of paper if known,'' unpublished.
\bibitem{b5} R. Nicole, ``Title of paper with only first word capitalized,'' J. Name Stand. Abbrev., in press.
\bibitem{b6} Y. Yorozu, M. Hirano, K. Oka, and Y. Tagawa, ``Electron spectroscopy studies on magneto-optical media and plastic substrate interface,'' IEEE Transl. J. Magn. Japan, vol. 2, pp. 740--741, August 1987 [Digests 9th Annual Conf. Magnetics Japan, p. 301, 1982].
\bibitem{b7} M. Young, The Technical Writer's Handbook. Mill Valley, CA: University Science, 1989.
\end{thebibliography}
\vspace{12pt}
\color{red}
IEEE conference templates contain guidance text for composing and formatting conference papers. Please ensure that all template text is removed from your conference paper prior to submission to the conference. Failure to remove the template text from your paper may result in your paper not being published.

\end{document}
