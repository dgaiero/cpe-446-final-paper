\documentclass[journal]{IEEEtran}
\IEEEoverridecommandlockouts
\usepackage{cite}
\usepackage{amsmath,amssymb,amsfonts}
\usepackage{algorithmic}
\usepackage{graphicx}
\graphicspath{{./design/pdf-snapshots/cropped/}{./assets/}{./design/video-output/noise-identification/}{./design/video-output/noisy-image/}{./design/video-output/post-nr/}{./design/video-output/reference/}}
\usepackage{textcomp}
\usepackage{xcolor}
\usepackage{float}
\usepackage{grffile}
\usepackage{hyperref}
\usepackage{lipsum}
\ifCLASSOPTIONcompsoc
\usepackage[caption=false, font=normalsize, labelfont=sf, textfont=sf]{subfig}
\else
\usepackage[caption=false, font=footnotesize]{subfig}
\fi
\def\BibTeX{{\rm B\kern-.05em{\sc i\kern-.025em b}\kern-.08em
    T\kern-.1667em\lower.7ex\hbox{E}\kern-.125emX}}
    
\newcommand{\textregisteredmark}{\textsuperscript{\tiny\textregistered}}
\renewcommand{\figureautorefname}{Fig.}
\newcommand{\subfigureautorefname}{\figureautorefname}

\begin{document}

\title{Correction Techniques for Image Faults}

\author{Dominic~Gaiero,~\textit{\href{mailto:dgaiero@calpoly.edu}{dgaiero@calpoly.edu}}
    Nicholas Serres,~\textit{\href{mailto:nserres@calopoly.edu}{nserres@calpoly.edu}}}

\maketitle

\begin{abstract}
Faults in image sensors can lead to a large drop in image reliability and quality. Image sensors are one of the most susceptible components to faults, and is one critical components in a camera system. This paper explores fault detection and correction of an image sensor by analyzing a sequence of frames from the camera to identify permanent faults and correct both permanent and transient faults. This paper is focused on applications in extreme environments, but applies to wider fields. 
\end{abstract}

\section{Introduction}
\IEEEPARstart{I}{n} this paper the effectiveness of detecting and correcting radiation-induced errors in an image sensor is explored. This provides an alternative on past methods by correcting and detecting faults in the image processing module, rather than a hardware level redesign. While using software to mitigate the effects of radiation on sensors is not a permanent solution, the benefits of significant reduced costs and development make this solution  appealing for non-critical applications. This project targets uses in control cameras at a radiation plants or other radiation intensive conditions where fault mitigation is desired. This will allow systems that use cameras in these conditions to operate with higher reliability and provide higher fidelity imagery despite the operating conditions of the cameras. Additionally, this solution provides enhanced reporting to the end user, unlike traditional image processing techniques.
\section{CMOS Background}
%% Can you also try and add some paragraphs? You can use the \par command.

CMOS(complementary metal oxide semiconductor) sensors are one of the most commonly used modern methods of creating digital images in cameras.  \autoref{fig:filters} illustrates how light passing from the outside passes through a glass lens and a color filter before finally reaching the photo-diodes.
\begin{figure}[H]
    \includegraphics[scale= .5]{assets/ImageSensor.png}
    \caption{CMOS Image Sensor}
    \label{fig:filters}
\end{figure}
The photo-diode is sensitive to light and is able to generate a current by being exposed to photons.  Finally that signal is sent to a central processor and translated into a pixel.(Tokyo Electron Museum)  CMOS sensors are very vulnerable to faults due to radiation and other interference because in some way the circuit has to be exposed to incoming rays and cannot be fully insulated. \autoref{fig:CMOS} shows a basic schematic for a CMOS sensor.  The photo diode has to be exposed to the incoming light so that it can activate the circuit, however, this makes it vulnerable to any other interference that can potentially harm the sensor.  
\begin{figure}[H]
    \includegraphics[scale= .75]{assets/Cmos schemateic.png}
    \caption{Basic CMOS Active Pixel}
    \label{fig:CMOS}
\end{figure}
One fault that often occurs in CMOS sensors is a Dark Current.  A dark current is when the CMOS circuit is activated by a wave that is not a photon.  In space and other radiation intensive environments this is a very relevant limitation of the technology because extreme accuracy is required with very high consistency.  Radiation that reaches a sensor is likely to cause the sensor to have some kind of fault, and if the intensity of the radiation is high enough then a pixel can become permanently damaged.  These Dark Currents when realized through the circuit and the central processor would appear to be a noisy picture or a collection of white pixels.  These faults can be harmless at first, but given enough time to build up they can render a sensor useless.
\section{Dark Currents}
Dark current is a very large limiting factor in the performance of an image sensor.  Image sensors work by generating a signal when an area of the signal has a different light level than the rest and it is able to then generate a pixel.  Dark currents "occur in every pixel, but vary from pixel-to-pixel" based on a number of factors such as heat on the image sensor.  Due to how pervasive dark currents are for image sensors there is a lot of research going into how to possibly prevent any dark currents.  Over the past 40 years there has  been nearly a 5000x drop in the dark current for image sensors, but this is still too high for many applications of image sensors.(Dan McGrath 2018)  Star trackers on sat elites need incredibly precise images to be taken on star trackers so that they can accurately calculate their position.  Any noise or dark currents induced in these sensors can results in large miscalculations.  Because some stars are so small relative to the star tracker, any noise and image can result in a failure of the star tracker.  In cases like star trackers on satellites dark current is a very dangerous fault.  
\section{Related Work} 
\input{relaedWorks}
\section{Method}
\subsection{Tools Used}
The model was designed using MATLAB\textregisteredmark\ and SIMULINK\textregisteredmark\ R2018B. Analysis was also performed using this software. Additionally, to aid testing, a generic video file was used provided by \hyperlink{https://blogs.unity3d.com/2016/11/28/free-vfx-image-sequences-flipbooks/}{Unity3D}.
\subsection{Assumptions}
The video file used has dimensions of 400x400px. Inside the parameters of certain blocks in the SIMULINK\textregisteredmark\ file this assumption was used and the model will need to be altered to work with video files of other dimensions. Additionally, the input video file was 30fps which is assumed in the frame counter module.
\subsection{Design}
The top level design is shown in Fig. \ref{fig:sysSpecs}. A general overview of this design is presented in this section with in-depth descriptions in later sections.
\begin{figure}[H]
    \includegraphics[width=\linewidth]{impl_dsgn}
    \caption{Top Level System Design}
    \label{fig:sysSpecs}
\end{figure}
\par An input video is fed into the \verb!From Multimedia File! block. Each color frame is converted to black and white to make image processing easier. Additionally, the \verb!EOF! output of the \verb!From Multimedia File! is fed into a frame counter's reset port, which is reset every time the video is restarted.
\par From this point, the video is sent to a noise generation unit. Since this model is a proof of concept, it is important to have a reference video as well as a video file with noise added. The noise generation block adds both permanent and transient noise.
\par After the frame has noise added, it is sent to two blocks, one to identify permanent noise in the frame, and another to filter out the noise in the frame.
\par Post noise generation, noise filtering, noise identification, and the reference frames are output to a video display as well as logged to a video file throughout simulation. Frame count information as well as other diagnostic and verification information is overlayed on top of the video frames to aid in analysis.
\subsubsection{ITU-R BT.709 Submodule}
The black and white conversion module follows the ITU-R BT.709 standard for color to black and white conversion. A reproduction of this module along with the coefficients used for each channel is shown in Fig. \ref{fig:btu709}.
\begin{figure}[H]
    \includegraphics[width=\linewidth]{impl_dsgn_ITU-R BT.709}
    \caption{ITU-R BT.709 Color to Black and White Conversion}
    \label{fig:btu709}
\end{figure}
\subsubsection{Noise Generator}
The Noise generator generates both permanent and transient noise.
\begin{figure}[ht!]
    \subfloat[Top Level Noise Generator Block\label{fig:1a:topLevelNoiseGenerator}]{%
       \includegraphics[width=\linewidth]{impl_dsgn_Noise Generator}}
    \\
     \subfloat[Permanent Noise Generator Block\label{fig:1b:permanentNoiseGenerator}]{%
        \includegraphics[width=\linewidth]{impl_dsgn_Noise Generator_Permanent Noise}}
    \caption{(a), (b) Shown are the top level noise generator block and the permanent noise generation block. The transient noise generation block is visually identical to the permanent noise generation block. The difference is inside of the MATLAB\textregisteredmark\ function block, and therefore was omitted.}
    \label{fig:noiseGenerator}
\end{figure}
\par As shown in Fig. \ref{fig:1a:topLevelNoiseGenerator}, the reference video frame is fed into the permanent noise generation block and then is fed into the transient noise generation block. Both the transient noise and permanent noise overlays are calculated each frame which results in an decrease in real-time performance of the demo. The transient noise generation block accounted for approximately 67\% of the total execution time followed next by the permanent noise generation block at approximaetly 3.2\% of the total execution time.
\section{Results}
This section is split into sections describing the verification process for each module and an analysis and discussion of the overall results from the simulation.
\subsection{Verification}
\par Noise Identification Verification is done by comparing the current frame to the previous frame. The percent difference is captured and overlaid onto the output video as shown in Fig. \ref{fig:noiseIDVer}. A one unit delay block is fed as the \verb!Test! input of this block.
 \begin{figure}[H]
    \includegraphics[width=\linewidth]{impl_dsgn_Noise Identification Verification}
    \caption{Noise Identification Verification}
    \label{fig:noiseIDVer}
\end{figure}

\par This module is verified using the peak signal-to-noise-ratio compared to the reference image. This is done in the \verb!Verification! block. The PSNR is calculated using the mean-square-error or MSE. The MSE represents the amount the reference image differs from the test image, and PSNR is a measure of the peak error and is calculated by dividing the maximum data type range by the MSE. This document uses floating point, therefore the maximum range is one\cite{mathworks}.
\section{Conclusion}
\par Further work can be done to produce HDL code or package the SIMULINK\textregisteredmark\ model into an IP package for deployment on an FPGA. Modules in Fig. \ref{fig:sysSpecs} with a yellow background color 

\bibliographystyle{IEEEtran}
{\footnotesize
\bibliography{IEEEabrv,refs}
}

\end{document}