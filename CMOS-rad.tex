\documentclass[journal]{IEEEtran}
\IEEEoverridecommandlockouts
% The preceding line is only needed to identify funding in the first footnote. If that is unneeded, please comment it out.
\usepackage{cite}
\usepackage{amsmath,amssymb,amsfonts}
\usepackage{algorithmic}
\usepackage{graphicx}
\graphicspath{{./design/pdf-snapshots/cropped/}{./assets/}{./design/video-output/}}
\usepackage{textcomp}
\usepackage{xcolor}
\usepackage{float}
\usepackage{grffile}
\usepackage{hyperref}
\usepackage{lipsum}
\ifCLASSOPTIONcompsoc
    \usepackage[caption=false, font=normalsize, labelfont=sf, textfont=sf]{subfig}
\else
\usepackage[caption=false, font=footnotesize]{subfig}
\fi
\def\BibTeX{{\rm B\kern-.05em{\sc i\kern-.025em b}\kern-.08em
    T\kern-.1667em\lower.7ex\hbox{E}\kern-.125emX}}
    
\newcommand{\textregisteredmark}{\textsuperscript{\tiny\textregistered}}

\begin{document}

\title{Correction Techniques for Image Faults}

% \author{\IEEEauthorblockN{1\textsuperscript{st} Dominic Gaiero}
% \IEEEauthorblockA{\textit{Computer Engineering Department} \\
% \textit{California Polytechnic State University, San Luis Obispo}\\
% San Luis Obispo, USA \\
% dgaiero@calpoly.edu}
% \and
% \IEEEauthorblockN{2\textsuperscript{nd} Given Name Surname}
% \IEEEauthorblockA{\textit{Computer Engineering Department} \\
% \textit{California Polytechnic State University, San Luis Obispo}\\
% San Luis Obispo, USA \\
% email address or ORCID}
% }

\author{Dominic~Gaiero,~\textit{\href{mailto:dgaiero@calpoly.edu}{dgaiero@calpoly.edu}}
        John~Doe,~\textit{\href{mailto:jdoe@email.com}{jdoe@email.com}}}

\maketitle
%Related Work
%Radiation Background
%Conclusion
%References
\begin{abstract}
Faults in image sensors can lead to a large drop in image reliability and quality.  The image sensor is one of the most susceptible components to faults, and is one critical components in a camera system.  This paper explores fault correction of an image sensor through passing the image through a filter to correct the image and potentially identify transient and permanent faults.  This solution is tailored for applications where radiation is present, but has applications in wider fields.  
\end{abstract}



\begin{IEEEkeywords}
CMOS, CIS
\end{IEEEkeywords}

\section{Introduction}%% Can you also try and add some paragraphs? You can use the \par command.
\IEEEPARstart{I}{n} this project the effectiveness radiation hardening a device through software was tested.  This expands on past methods by having picture correction happen in the data processing phase, rather than redesigning from a hardware level. While using software to mitigate the effects of radiation on sensors is not a permanent solution, the benefits of significant reduced costs and development make this solution very appealing for less critical applications. This project targets uses in security at a radiation plant or non-critical cameras in space or other radiation intensive conditions. This will allow systems that use cameras in these conditions to operate with more reliability and allow for more clear images despite the conditions and operation of the cameras.  Also, this solution aims to give more information on the nature of a fault in a sensor to a user, unlike traditional image processing techniques.

\section{CMOS Background}
%% Can you also try and add some paragraphs? You can use the \par command.

CMOS(complementary metal oxide semiconductor) sensors are one of the most commonly used modern methods of creating digital images in cameras.  \autoref{fig:filters} illustrates how light passing from the outside passes through a glass lens and a color filter before finally reaching the photo-diodes.
\begin{figure}[H]
    \includegraphics[scale= .5]{assets/ImageSensor.png}
    \caption{CMOS Image Sensor}
    \label{fig:filters}
\end{figure}
The photo-diode is sensitive to light and is able to generate a current by being exposed to photons.  Finally that signal is sent to a central processor and translated into a pixel.(Tokyo Electron Museum)  CMOS sensors are very vulnerable to faults due to radiation and other interference because in some way the circuit has to be exposed to incoming rays and cannot be fully insulated. \autoref{fig:CMOS} shows a basic schematic for a CMOS sensor.  The photo diode has to be exposed to the incoming light so that it can activate the circuit, however, this makes it vulnerable to any other interference that can potentially harm the sensor.  
\begin{figure}[H]
    \includegraphics[scale= .75]{assets/Cmos schemateic.png}
    \caption{Basic CMOS Active Pixel}
    \label{fig:CMOS}
\end{figure}
One fault that often occurs in CMOS sensors is a Dark Current.  A dark current is when the CMOS circuit is activated by a wave that is not a photon.  In space and other radiation intensive environments this is a very relevant limitation of the technology because extreme accuracy is required with very high consistency.  Radiation that reaches a sensor is likely to cause the sensor to have some kind of fault, and if the intensity of the radiation is high enough then a pixel can become permanently damaged.  These Dark Currents when realized through the circuit and the central processor would appear to be a noisy picture or a collection of white pixels.  These faults can be harmless at first, but given enough time to build up they can render a sensor useless.

\section{Related Work} 


\section{Method}
\subsection{Tools Used}
The model was designed using MATLAB\textregisteredmark\ and SIMULINK\textregisteredmark\ R2018B. Analysis was also performed using this software. Additionally, to aid testing, a generic video file was used provided by \hyperlink{https://blogs.unity3d.com/2016/11/28/free-vfx-image-sequences-flipbooks/}{Unity3D}.
\subsection{Assumptions}
The video file used has dimensions of 400x400px. Inside the parameters of certain blocks in the SIMULINK\textregisteredmark\ file this assumption was used and the model will need to be altered to work with video files of other dimensions. Additionally, the input video file was 30fps which is assumed in the frame counter module.
\subsection{Design}
The top level design is shown in Fig. \ref{fig:sysSpecs}. A general overview of this design is presented in this section with in-depth descriptions in later sections.
\begin{figure}[H]
    \includegraphics[width=\linewidth]{impl_dsgn}
    \caption{Top Level System Design}
    \label{fig:sysSpecs}
\end{figure}
\par An input video is fed into the \verb!From Multimedia File! block. Each color frame is converted to black and white to make image processing easier. Additionally, the \verb!EOF! output of the \verb!From Multimedia File! is fed into a frame counter's reset port, which is reset every time the video is restarted.
\par From this point, the video is sent to a noise generation unit. Since this model is a proof of concept, it is important to have a reference video as well as a video file with noise added. The noise generation block adds both permanent and transient noise.
\par After the frame has noise added, it is sent to two blocks, one to identify permanent noise in the frame, and another to filter out the noise in the frame.
\par Post noise generation, noise filtering, noise identification, and the reference frames are output to a video display as well as logged to a video file throughout simulation. Frame count information as well as other diagnostic and verification information is overlayed on top of the video frames to aid in analysis.
\subsubsection{ITU-R BT.709 Submodule}
The black and white conversion module follows the ITU-R BT.709 standard for color to black and white conversion. A reproduction of this module along with the coefficients used for each channel is shown in Fig. \ref{fig:btu709}.
\begin{figure}[H]
    \includegraphics[width=\linewidth]{impl_dsgn_ITU-R BT.709}
    \caption{ITU-R BT.709 Color to Black and White Conversion}
    \label{fig:btu709}
\end{figure}
\subsubsection{Noise Generator}
The Noise generator generates both permanent and transient noise.
\begin{figure}[ht!]
    \subfloat[Top Level Noise Generator Block\label{fig:1a:topLevelNoiseGenerator}]{%
       \includegraphics[width=\linewidth]{impl_dsgn_Noise Generator}}
    \\
     \subfloat[Permanent Noise Generator Block\label{fig:1b:permanentNoiseGenerator}]{%
        \includegraphics[width=\linewidth]{impl_dsgn_Noise Generator_Permanent Noise}}
    \caption{(a), (b) Shown are the top level noise generator block and the permanent noise generation block. The transient noise generation block is visually identical to the permanent noise generation block. The difference is inside of the MATLAB\textregisteredmark\ function block, and therefore was omitted.}
    \label{fig:noiseGenerator}
\end{figure}
\par As shown in Fig. \ref{fig:1a:topLevelNoiseGenerator}, the reference video frame is fed into the permanent noise generation block and then is fed into the transient noise generation block. Both the transient noise and permanent noise overlays are calculated each frame which results in an decrease in real-time performance of the demo. The transient noise generation block accounted for approximately 67\% of the total execution time followed next by the permanent noise generation block at approximaetly 3.2\% of the total execution time.
% \section{

\bibliographystyle{IEEEtran}
{\footnotesize
\bibliography{IEEEabrv,refs}
}

\end{document}