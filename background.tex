%% Can you also try and add some paragraphs? You can use the \par command.

CMOS(complementary metal oxide semiconductor) sensors are one of the most commonly used modern methods of creating digital images in cameras.  They work by having light filter into a light sensitive diode that is then able to activate a transistor so that a signal can be generated.  (Insert photo of CMOS circuit)  The photo-diode is sensitive to light and is able to generate a current by being exposed to photons.  Finally that signal is sent to a central processor and translated into a pixel.  CMOS sensors are very vulnerable to faults due to radiation and other interferences because in some way the circuit has to be exposed and cannot be fully insulated.  One fault that often occurs in CMOS sensors is a Dark Current.  A Dark Current is when the CMOS circuit is activated by a wave that is not a photon.  In space and other radiation intensive environments this is a very relevant limitation of the technology because extreme accuracy is required with very high consistency.  Radiation that reaches a sensor is likely to cause the sensor to have some kind of fault, and if the intensity of the radiation is high enough then a pixel can become permanently damaged.  These Dark Currents when realized through the circuit and the central processor would appear to be a noisy picture or a collection of white pixels.  These faults can be harmless at first, but given enough time to build up they can render a sensor useless.