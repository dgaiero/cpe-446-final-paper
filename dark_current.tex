Dark current is a very large limiting factor in the performance of an image sensor.  Image sensors work by generating a signal when an area of the signal has a different light level than the rest and it is able to then generate a pixel.  Dark currents "occur in every pixel, but vary from pixel-to-pixel" based on a number of factors such as heat on the image sensor.  Due to how pervasive dark currents are for image sensors there is a lot of research going into how to possibly prevent any dark currents.  Over the past 40 years there has  been nearly a 5000x drop in the dark current for image sensors, but this is still too high for many applications of image sensors.(Dan McGrath 2018)  Star trackers on sat elites need incredibly precise images to be taken on star trackers so that they can accurately calculate their position.  Any noise or dark currents induced in these sensors can results in large miscalculations.  Because some stars are so small relative to the star tracker, any noise and image can result in a failure of the star tracker.  In cases like star trackers on satellites dark current is a very dangerous fault.  